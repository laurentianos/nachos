\documentclass[12pt, letterpaper]{report}
\usepackage{varioref}
\usepackage[boxruled]{algorithm2e} %for psuedo code
\SetKwInOut{Parameter}{parameter}
%\begin{algorithm}[H]
%  \caption{TK caption}
%    \SetAlgoLined
%    \SetKwInOut{Input}{input}
%    \SetKwInOut{Output}{output}
%    \Input{A bitmap $Im$ of size $w\times l$}
%    \Output{A partition of the bitmap}
%    \KwData{this text}
%    \KwResult{how to write algorithm with \LaTeX2e }
%    initialization\;
%    \While{not at end of this document}{
%      read current\;
%      \eIf{understand}{
%        go to next section\;
%        current section becomes this one\;
%        }{
%        go back to the beginning of current section\;
%      }
%    }
%\end{algorithm}

\newcommand{\code}[1]{\texttt{#1}}
\labelformat{algocf}{\textit{Algorithm}\,#1}

\begin{document}
\title{Design Document for Nachos \\
       Phase 1: Building a Thread System}
\author{\textbf{Abdulaziz Awliyaa, 
        Christopher Connors,
        Brian Giroux} \\
        COSC 3407 - Operating Systems \\
        Group 1}
\maketitle

\section*{Task I: Implement \code{KThread} Joining}

In \code{KThread}, we implement the \code{join()} method. When called, this method will cause the the calling thread to wait on another thread before continuing.

\subsection*{Implementing the \code{join()} Method}

The \code{join()} method requests access to the CPU, it checks to make sure nothing is running; if it is, it will wait on the queue.

If \code{join()} is called a second time, the behaviour is undefined

See \vref{method_join}.

\begin{algorithm}[!ht]
  \caption{The \code{join()} method}
  \label{method_join}
    \SetAlgoLined
    \If{this thread is the current thread}{abandon\;}
    disable interrupts\;
    \If{this thread is running}{
      put this thread on the wait queue\;
      put this thread to sleep\;
    }
    enable interrupts\;
\end{algorithm}


\subsection*{Test Cases for Task I}

\begin{enumerate}
\item We will create a thread
\item Start a second thread
\item We will try to join one of the threads to itself to make sure it can't join itself
\item We will join one thread to the other thread
\item We will verify that the thread waits for the first thread
\end{enumerate}

\section*{Task II: Implementing the \code{Condition2} Class}

In the \code{Condition2} class, we implement three methods: \code{sleep()}, \code{wake()} and \code{wakeAll()}.

\subsection*{Implementing the \code{sleep()} Method}

First we have to make sure that the current thread holds the lock. Then we atomically add the thread to the queue and put it to sleep.

We must release the lock before putting the thread to sleep otherwise no other thread can acquire the lock.

See \vref{method_sleep}.

\begin{algorithm}[!ht]
  \caption{The \code{sleep()} method}
  \label{method_sleep}
    \SetAlgoLined
    \If{the current thread does not hold the lock}{abandon\;}
    disable interrupts\;
    add the current thread to the wait queue\;
    release the lock\;
    put the thread to sleep\;
    acquire the lock\;
    enable the interrupts\;
\end{algorithm}

\subsection*{Implementing the \code{wake()} Method}

We check to make sure that the current thread has the lock. Then, we atomically check the wait queue and if there is a thread on the wait queue, wake it up.

See \vref{method_wake}.

\begin{algorithm}[!ht]
  \caption{The \code{wake()} method}
  \label{method_wake}
    \SetAlgoLined
    \If{the current thread does not hold the lock}{abandon\;}
    disable interrupt\;
    \If{the first item on the readyQueue is a KThread}{
      call KThread.ready();
    }
    enable interrupts\;
\end{algorithm}

\subsection*{Implementing the \code{wakeAll()} Method}

We simply call \code{wake()} on the first thread on the wait queue until the wait queue is empty.

See \vref{method_wakeall}.

\begin{algorithm}[!ht]
  \caption{The \code{wakeAll()} method}
  \label{method_wakeall}
    \SetAlgoLined
    disable interrupts\;
    \If{readyQueue is not empty}{
      \ForEach{KThread in the readyQueue}{
        wake the KThread\;
      }
    }
    enable interrupts
\end{algorithm}

\subsection*{Test Cases for Task II}

\begin{enumerate}
\item We will create some threads (5 or more) putting them to sleep as we go along
\item We will call \code{wake()} to test that one thread wakes up
\item We will call \code{wakeAll()} to make sure that the rest of the threads wake up.
\end{enumerate}

\section*{Task III: Implementing \code{Alarm}}

In this section we implement the \code{timerInterrupt()} and \code{waitUntil()} methods.

\subsection*{Implementing the \code{timerInterrupt()} Method}

The \code{timerInterrupt()} method will wake the first item on the wait queue if its time is up. The wait queue will be a priority queue and the items on the wait queue will contain a thread and a wake time.

See \vref{method_timerinterrupt}.

\begin{algorithm}[!ht]
  \caption{The \code{timerInterrupt()} method}
  \label{method_timerinterrupt}
    \SetAlgoLined
    disable interrupts\;
    \If{the readyQueue is not empty}{
      check the wake time for the thread at the head of the queue\;
      if it's time for him to wake up the wake him up\;
    }
    enable interrupts\;
    yield to the next thread\;
\end{algorithm}

\subsection*{Implementing the \code{waitUntil()} Method}

This method will calculate the threads wake time, then put the thread along with its wake time on the waitQueue.

See \vref{method_waituntil}.

\begin{algorithm}[!ht]
  \caption{The \code{waitUntil()} method}
  \label{method_waituntil}
  \SetAlgoLined
  \SetKwInOut{Input}{input}
  \Input{the amout of time ($x$) to wait}
    disable interrupts\;
    set wake time to current time + $x$\;
    put the thread and wake time onto the waitQueue (priority queue)\;
    go to sleep\;
    enable interrupts\;
\end{algorithm}

\subsection*{Test Cases for Task III}

We will make some threads and put them to sleep for various amounts of time. We will watch as they are removed from the wait queue.


\section*{Task IV: Implementing \code{Communicator}}

The \code{Communicator} class allows threads to communicate. We will implement two methods: the \code{speak()} and the \code{listen()} methods.

\subsection*{The \code{speak()} Method}

Speakers wait until there are no listeners or other speakers before writing to the variable. Then, they wake up a waiting listener.

See \vref{method_speak}.

\begin{algorithm}[!ht]
  \caption{The \code{speak()} method}
  \label{method_speak}
    \SetAlgoLined
    \SetKwInOut{Input}{input}
    \SetKwInOut{Output}{output}
    \Input{an integer value to speak}
  acquire lock\;
    \eIf{the listen queue is empty}{
      create a new speaker in the speaker queue\;
      set its word to the spoken word\;
      put the speaker to sleep\;
    }{
      get the first listener off the listen queue\;
      set the listener word to the spoken word\;
      wake up the listener\;
    }
  release lock\;
\end{algorithm}

\subsection*{The \code{listen()} Method}

Listeners wait until there are no speakers to read the variable. Then, they wake up a waiting speaker.

See \vref{method_listen}.

\begin{algorithm}[!ht]
  \caption{the \code{listen()} method}
  \label{method_listen}
    \SetAlgoLined
    \SetKwInOut{Input}{input}
    \SetKwInOut{Output}{output}
    \Output{the integer word spoken by the speaker}
    acquire lock\;
    \eIf{the speaker queue is empty}{
      create a new listener in the listener queue\;
      put the listener to sleep\;
      get the sleeper's word\;
    }{
      remove a speaker from the queue\;
      get the speaker's word\;
      wake up the speaker\;
     }
    release the lock\;
    return the word\;
\end{algorithm}


\subsection*{Test Cases for Task IV}

\begin{enumerate}
\item Speak some words (4 or 5)
\item Call some listeners (2 or 3)
\item Make sure that each listener heard what was spoken
\item Check the speaker queue to make sure that they've been removed
\end{enumerate}

\section*{Task V: Implementing \code{ReactWater}}

We will keep two global variables: $H$ to count the number of hydrogen atoms present and $O$ to count the number of oxygen atoms. The \code{hReady()} and \code{oReady()} methods will increment $H$ and $O$ respectively and the \code{makeWater()} method decrement the variables by the appropriate amounts and display a message.

\subsection*{The \code{ReactWater} Constructor}

The constructor will simply initialize the $H$ and $O$ variables to $0$.

See \vref{method_reactwater}.

\begin{algorithm}[!ht]
  \caption{the \code{ReactWater} constructor}
  \label{method_reactwater}
    \SetAlgoLined
    set $H$ and $O$ to $0$\;
\end{algorithm}

\subsection*{The \code{hReady()} Method}

\code{hReady()} will increment $H$ by $1$ and call \code{makeWater()}.

See \vref{method_hready}.

\begin{algorithm}[!ht]
  \caption{the \code{hReady()} method}
  \label{method_hready}
    \SetAlgoLined
    increment $H$ by $1$\;
    call make water\;
\end{algorithm}

\subsection*{The \code{oReady()} Method}

\code{oReady()} will increment O by 1 and call \code{makeWater()}.

See \vref{method_oready}.

\begin{algorithm}[!ht]
  \caption{the \code{oReady()} method}
  \label{method_oready}
    \SetAlgoLined
     increment $O$ by $1$\;
     call make water\;
\end{algorithm}

\subsection*{The \code{makeWater()} Method}

\code{makeWater()} will check how many of $H$ and $O$ we have. If we have sufficient quantities, we will decrement $H$ by $2$ and $O$ by $1$ and print a message.

See \vref{method_makewater}.

\begin{algorithm}[!ht]
  \caption{the \code{makeWater()} method}
  \label{method_makewater}
    \SetAlgoLined
    \If{$H \geq 2$ and $O \geq 1$}{
      decrement $H$ by $2$\;
      decrement $O$ by $1$\;
      print the "I made water" message\;
    }
\end{algorithm}
\end{document}
