\documentclass[12pt, letterpaper]{report}
\usepackage{amsmath}
\usepackage{varioref}
\usepackage{listings}
\usepackage[boxruled]{algorithm2e} % for psuedo code
\SetKwInOut{Parameter}{parameter}
\SetKwInOut{Return}{return}
\newcommand{\code}[1]{\texttt{#1}}
\labelformat{algocf}{\textit{Algorithm}\,#1}

\begin{document}
\title{Design Document for Nachos \\
       Phase 1: Multiprogramming}
 \author{\textbf{Abdulaziz Awliyaa,
                 Christopher Connors,
                 Brian Giroux} \\
         COSC 3407 - Operating Systems \\
         Group 1}
\maketitle

\section*{Task I -- Implement File System Calls}

%#############UserKernel.java########################
\subsection*{Changes to \code{UserKernel.java} and Creation of New \code{File} class}
In \code{UserKernel.java}, we will create a 64 element array which will store all global \code{File} file objects which are currently active. The structure of the \code{File} class will look like this:

\lstset{language=Java,caption={Outline of \code{File} class},label=FileClass}
\begin{lstlisting}
public class File{
  Object OpenFile
  int counter = 1;
  boolean isWriting = FALSE;
  boolean isLinked = TRUE;

  /* ACCESSORS */

  /**
   * @return OpenFile object
   */
  OpenFile getOpenFile();

  /**
   * @return this.counter
   */
  int getCounter();

  /**
   * @return TRUE if this.counter is greater than zero
   *         (ie, there is at least one open file)
   */
  boolean isOpen();

  /**
   * @return this.isWriting
   */
  boolean getIsWriting();

  /**
   * @return this.isLinked
   */
  boolean getIsLinked();


  /* MUTATORS */

  /**
   * increments the this.counter by 1
   */
  void incCounter();

  /**
   * decrement the this.counter by 1
   */
  void decCounter();

  /**
   * Sets this.isWriting to TRUE.
   */
  void setWriting(boolean writing);

  /**
   * Sets this.isLinked to TRUE.
   */
  void setIsLinked(boolean linked);
}
\end{lstlisting}


%#################UserProcess.java###########################
\subsection*{Changes to the \code{UserProcess.java} class}
\subsubsection*{Implementing the \code{creat()} Method}
This method will create a new file if it does not already exist.
If it was already created (ie, it exists in the global array),
we will just link to it.

See \vref{method_creat}.

\begin{algorithm}[!ht]
  \caption{The \code{creat()} method}
  \label{method_creat}
  \SetAlgoLined
  \Parameter{*name -- a pointer to the filename}
  \Return{int}
  \BlankLine
  \eIf{file is in global array of files}{
    \If{we are not already linked to the file}{
      in user process array, add link to the file in global array\;
    }
  }{
    in global array, add the file\;
    in user process array, add link to the file in global array\;
  }
\end{algorithm}


\subsubsection*{Implementing the \code{open()} Method}
We can only open an existing file.
If a file does not exist, we return ${-1}$,
otherwise it returns the file descriptor.

See \vref{method_open}.

\begin{algorithm}[!ht]
  \caption{The \code{open()} method}
  \label{method_open}
  \SetAlgoLined
  \Parameter{*name -- pointer to the name of the file to open}
  \Return{int -- the file descriptor of the file}
  \BlankLine
  \eIf{OpenFile array exists}{
    \eIf{the file is in the array}{
      call setIsOpen\;
      set TRUE\;
      \KwRet{file descriptor (local index)}\;
    }{
      \eIf{the file is in the global array}{
        \eIf{the file is linked}{
          in user process array, add link to the file in global array\;
          \KwRet{file descriptor (local index)}\;
        }{
          \KwRet{${-1}$}\;
        }
      }{
        \KwRet{${-1}$}\;
      }
    }
  }{
    \KwRet{${-1}$}\;
  }
\end{algorithm}


\subsubsection*{Implementing the \code{read()} Method}
This method will allow us to read a number of bytes from a file.
The bytes we read from the file will be placed in a buffer.

See \vref{method_read}.

\begin{algorithm}[!ht]
  \caption{The \code{read()} method}
  \label{method_read}
  \SetAlgoLined
  \Parameter{int fileDescriptor -- the file descriptor of the file to read\\
             *buffer[] -- a pointer to the buffer array\\
             int count -- the number of bytes to read}
  \Return{int -- the number of bytes actually read}
  \BlankLine
  $counter \gets 0$\;
  \eIf{the file descriptor is valid}{
    \If{the file is open}{
      \While{not end of file}{
        read byte from file to buffer\;
        increment counter\;
      }
      \KwRet{counter}\;
    }
  }{
    \KwRet{${-1}$}\;
  }
\end{algorithm}


%Method int write(int fileDescriptor, void *buffer, int count)
\subsubsection*{Implementing the \code{write()} Method}
This method will allow us to write a number of bytes from to file.
The bytes we write to the file will come from a buffer.

See \vref{method_write}.

\begin{algorithm}[!ht]
  \caption{The \code{write()} method}
  \label{method_write}
  \SetAlgoLined
  \Parameter{int fileDescriptor -- the file descriptor of the file to write to\\
             *buffer[] -- a pointer to the buffer array\\
             int count -- the number of bytes to write}
  \Return{int -- the number of bytes written}
  \BlankLine
  \eIf{the file descriptor is valid}{
    \If{file is open}{
      $isWriting \gets \text{TRUE}$\;
      \For{$i \gets 0$ \KwTo $count - 1$}{
        write buffer[i] to file\;
      }
      $isWriting \gets \text{FALSE}$\;
    }
  }{
    \KwRet{${-1}$}\;
  }
\end{algorithm}


%Method int close(int fileDescriptor)
\subsubsection*{Implementing the \code{close()} Method}
This method closes a file if it is not being written to by another process.

See \vref{method_close}.

\begin{algorithm}[!ht]
  \caption{The \code{close()} method}
  \label{method_close}
  \SetAlgoLined
  \Parameter{int fileDescriptor -- the file descriptor of the file to close}
  \Return{int -- returns ${-1}$ if there was a problem}
  \BlankLine
  \eIf{the file descriptor is valid}{
    \eIf{the file is open}{
      \If{the file is being written to}{
        wait for the write to end\;
      }
      decrement the files counter\;
      call unlink()\;
    }{
      \KwRet{${-1}$}\;
    }
  }{
    \KwRet{${-1}$}\;
  }
\end{algorithm}


\subsubsection*{Implementing the \code{unlink()} Method}
After a process closes a file, it must be unlinked from the table.
See \vref{method_unlink}.

\begin{algorithm}[!ht]
  \caption{The \code{unlink()} method}
  \label{method_unlink}
  \SetAlgoLined
  \Parameter{*name -- pointer to the name of the file to unlink}
  \BlankLine
  \If{file exists in global array}{
    \eIf{file is open}{
      set isLinked to FALSE\;
    }{
      remove file from global array\;
    }
  }
\end{algorithm}


\subsection*{Test Cases}
We will:
\begin{itemize}
\item open some files
\item close the files
\item unlink them then try to re-open them
\item call halt from a running process that that has open files
\item read and write to open files
\item read and write to 1 and 0 fileDescriptor without opening them
\end{itemize}
\clearpage

\section*{Task II -- Implement Support for Multiprogramming}
\subsection*{Changes to \code{UserKernel.java}}
We will create a linked list to use as a global page table.
We will be able to call \code{getNumPhysPages()}
and set the max size of the Linked List to this number.
The head of the global page table will be set to the first available page.

\subsection*{Changes to \code{UserProcess.java}}
We will first check if pageTable is already created.
If it doesn't already exist,
we will create an array of 8 elements that will serve as the pageTable.

We will create a new array of type \code{TranslationEntry[]}
called \code{translations1}.
This pageTable will request free pages from the \code{UserKernel}
global page table and ask for the head of the list.
It will then load that page, move the head of the list,
load the next page, and so on, until all 8 pages requested
are placed in the array.
Then it will request that the kernel load in the program from memory.
It will use the \code{translations1} array to translate between the physical memory and the virtual memory.

When the program calls \code{exit()},
we must make sure that it sets the bit in the global array
to indicate that it is free,
and to move the head of the list.

To read and write virtual memory,
we must make sure we check the PID of the requesting process
and check which virtual pages it belongs to,
that way a process can not read from or write to another processes table.

\subsubsection*{Implementing \code{loadSection()}}
This method will allocate the pages that a process needs.

\subsection*{Test Cases}
To test this task, we will:
\begin{itemize}
\item load programs into memory
\item try and fill memory and load another process
\item load multiple programs into one process and see if the memory fails
\item close programs and make sure memory is freed
\item open 2 processes and try to read and write to each other's
\item page table which should not work
\end{itemize}
\clearpage


\section*{Task III -- Implement System Calls}
\subsection*{Implementing the \code{join()} Method}
Only a child can join a parent so we must make sure that when the parent
calls \code{join()},
it knows the PID of the created process and those underneath it.
We will keep a linked list of child PIDs in the process.

We will check that \code{childExit = FALSE}
to make sure that the child does indeed exist before joining it.

Inside the \code{UserKernel.java},
we will create a PID counter that counts up every time
a new process is created.

In \code{UserProcess.java},
we need need a field called ``\code{int pid}''
that is set by the kernel on startup.


\subsection*{Implementing the \code{exit()} Method}
We will set global page table to release
virtual memory and physical memory.
We will set PID and name to null (the process is no longer accessible).

The parent process will have a method called
\code{childExit()} that is called by the child when it exits
and sets a Boolean flag.
\end{document}
